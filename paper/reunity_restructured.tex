% ReUnity: A Trauma-Aware AI Framework for Identity Continuity Support
% Restructured for Publication - Main Body 25-30 Pages
% Copyright (c) 2024 Christopher Ezernack, REOP Solutions. All rights reserved.

\documentclass[12pt,a4paper]{article}

% Essential packages
\usepackage[utf8]{inputenc}
\usepackage[T1]{fontenc}
\usepackage{amsmath,amssymb,amsfonts}
\usepackage{graphicx}
\usepackage{booktabs}
\usepackage{hyperref}
\usepackage{xcolor}
\usepackage{float}
\usepackage{geometry}
\usepackage{fancyhdr}
\usepackage{titlesec}
\usepackage[backend=biber,style=numeric,sorting=none]{biblatex}
\usepackage{listings}
\usepackage{algorithm}
\usepackage{algpseudocode}

\geometry{margin=1in}
\addbibresource{reunity_complete.bib}

% Custom commands
\newcommand{\reunity}{\textsc{ReUnity}}

% Header/Footer
\pagestyle{fancy}
\fancyhf{}
\rhead{\reunity{} Framework}
\lhead{REOP Solutions}
\rfoot{Page \thepage}

\begin{document}

% Title Page
\begin{titlepage}
\centering
\vspace*{2cm}

\includegraphics[width=0.25\textwidth]{figures/reop_solutions_logo.png}\\[1cm]

{\huge\bfseries \reunity{}: A Trauma-Aware AI Framework for Identity Continuity Support}\\[0.5cm]
{\large An Entropy-Based Approach to Memory Reconstruction and Emotional State Analysis}\\[2cm]

{\Large Christopher Ezernack}\\[0.3cm]
{\large REOP Solutions}\\[0.3cm]
{\normalsize christopher@reopsolutions.com}\\[2cm]

{\large \today}\\[3cm]

\textit{Confidential: For Research and Development Purposes Only}\\[0.5cm]
\textcopyright{} 2024 Christopher Ezernack, REOP Solutions. All rights reserved.\\[1cm]

\textbf{DISCLAIMER:} This document describes a research framework and is not a clinical or treatment tool. It does not provide medical advice, diagnosis, or treatment. If you are in crisis, please contact emergency services or a crisis hotline.

\end{titlepage}

\newpage
\tableofcontents
\newpage

% ============================================================================
% MAIN BODY - Target: 25-30 Pages
% ============================================================================

\section{Introduction}
\label{sec:introduction}

Trauma-related identity fragmentation affects millions of individuals worldwide, manifesting in conditions such as Dissociative Identity Disorder (DID), Complex Post-Traumatic Stress Disorder (C-PTSD), and Borderline Personality Disorder (BPD). Current intervention systems consistently fail these populations due to geographic barriers, institutional gatekeeping, and approaches that prioritize surveillance over survivor autonomy.\footnote{Rural communities face particular challenges: Montana women experience intimate partner violence at 56.7\% lifetime prevalence, with 61\% of counties lacking trauma-informed psychiatric providers. See Appendix A for detailed statistics.}

The \reunity{} framework addresses these failures through a fundamentally different approach: an entropy-based AI system designed to support identity continuity during fragmentation episodes. Rather than attempting to ``fix'' or ``integrate'' fragmented states, \reunity{} serves as an external memory anchor that maintains awareness across emotional discontinuities.

\subsection{Problem Statement}

Individuals experiencing trauma-related fragmentation face three interconnected challenges:

\begin{enumerate}
\item \textbf{Memory Discontinuity:} Emotional amnesia creates gaps in autobiographical memory, preventing coherent self-narrative construction.
\item \textbf{Relational Pattern Blindness:} Fragmentation impairs the ability to recognize harmful relational dynamics across time.
\item \textbf{Institutional Failure:} Healthcare and support systems designed for neurotypical populations systematically exclude those with fragmented identity states.\footnote{See Appendix A for case studies documenting institutional failures in Montana State University and rural healthcare settings.}
\end{enumerate}

\subsection{Contributions}

This paper makes the following contributions:

\begin{enumerate}
\item A mathematical framework for emotional state analysis using Shannon entropy, Jensen-Shannon divergence, and Lyapunov stability measures.
\item The Recursive Identity Memory Engine (RIME), which maintains encrypted identity fragments with consent-based access controls.
\item A Protective Logic Module that detects harmful relational patterns without requiring conscious user recognition.
\item Empirical validation using the GoEmotions dataset (n=54,263) demonstrating the framework's analytical capabilities.
\item An open-source implementation with privacy-preserving design principles.
\end{enumerate}

\section{Related Work}
\label{sec:related}

\subsection{Trauma-Informed Technology}

Digital interventions for trauma survivors have evolved from simple journaling applications to sophisticated AI-powered systems. However, most existing approaches share a fundamental limitation: they assume stable identity states and continuous memory access.

Dialectical Behavior Therapy (DBT) applications provide skills training but lack mechanisms for supporting users during dissociative episodes. Crisis intervention chatbots offer immediate support but cannot maintain context across fragmented sessions. Electronic health records capture clinical data but fail to preserve the subjective experience of identity discontinuity.

\subsection{Information-Theoretic Approaches to Psychology}

The application of information theory to psychological phenomena has a rich history. Shannon entropy has been used to quantify emotional complexity, while mutual information networks have modeled cognitive dependencies. The Free Energy Principle provides a theoretical framework for understanding how biological systems minimize surprise through predictive modeling.

\reunity{} extends these approaches by applying entropy-based analysis specifically to identity fragmentation, treating emotional states as probability distributions whose dynamics can be mathematically characterized.

\subsection{Privacy-Preserving AI}

Federated learning enables collaborative model training without centralizing sensitive data. Differential privacy provides mathematical guarantees for privacy loss quantification. Homomorphic encryption allows computation on encrypted data. \reunity{} integrates these techniques to ensure that survivor data remains under survivor control.

\section{System Overview}
\label{sec:overview}

\reunity{} consists of seven interconnected modules designed to support identity continuity:

\begin{enumerate}
\item \textbf{EESA:} Entropy-Based Emotional State Analyzer
\item \textbf{RIME:} Recursive Identity Memory Engine
\item \textbf{PLM:} Protective Logic Module
\item \textbf{RCT:} Relationship Continuity Threader
\item \textbf{MLDC:} MirrorLink Dialogue Companion
\item \textbf{AAS:} Alter-Aware Subsystem
\item \textbf{CCI:} Clinician and Caregiver Interface
\end{enumerate}

\begin{figure}[H]
\centering
\includegraphics[width=0.9\textwidth]{figures/system_architecture_diagram.pdf}
\caption{ReUnity system architecture showing the seven core modules and their interactions. Data flows from user input through entropy analysis to memory storage and protective logic, with all operations governed by consent controls.}
\label{fig:architecture}
\end{figure}

The system operates on three fundamental principles:

\begin{enumerate}
\item \textbf{Survivor Autonomy:} Users maintain complete control over their data, including the ability to delete, export, or restrict access at any time.
\item \textbf{Non-Pathologizing Design:} Fragmentation is treated as an adaptive response to trauma, not a disorder to be corrected.
\item \textbf{Privacy by Default:} All data is encrypted locally; cloud synchronization is optional and user-controlled.
\end{enumerate}

\section{Methodology}
\label{sec:methodology}

\subsection{Shannon Entropy for Emotional State Analysis}

The core mathematical framework uses Shannon entropy to quantify emotional state fragmentation:

\begin{equation}
H(X) = -\sum_{i=1}^{n} p(x_i) \log_2 p(x_i)
\label{eq:shannon}
\end{equation}

where $H(X)$ represents the entropy of the emotional state distribution, $p(x_i)$ is the probability of emotional state $i$, and $n$ is the total number of discrete emotional states.

Higher entropy values indicate greater emotional variability, which may signal either healthy emotional flexibility or destabilizing fragmentation depending on context and rate of change.

\subsection{Jensen-Shannon Divergence for State Transitions}

To track transitions between emotional configurations, we employ the Jensen-Shannon divergence:

\begin{equation}
JSD(P||Q) = \frac{1}{2}D_{KL}(P||M) + \frac{1}{2}D_{KL}(Q||M)
\label{eq:jsd}
\end{equation}

where $M = \frac{1}{2}(P + Q)$ is the midpoint distribution and $D_{KL}$ denotes the Kullback-Leibler divergence.

The JSD is symmetric and bounded between 0 and 1, making it suitable for detecting dramatic shifts between emotional states without bias toward particular configurations.

\subsection{Lyapunov Stability Analysis}

System stability is assessed using Lyapunov exponents:

\begin{equation}
\lambda = \lim_{t \to \infty} \frac{1}{t} \sum_{i=0}^{t-1} \log_2 |f'(x_i)|
\label{eq:lyapunov}
\end{equation}

Positive Lyapunov exponents indicate chaotic dynamics (potential crisis states), while negative values indicate stable emotional trajectories. This allows the system to predict destabilization before it becomes clinically apparent.

\subsection{Mutual Information for Pattern Recognition}

Dependencies between emotional states and relational contexts are quantified using mutual information:

\begin{equation}
I(X;Y) = H(X) + H(Y) - H(X,Y)
\label{eq:mi}
\end{equation}

High mutual information between specific emotional states and relational patterns may indicate trauma bonds or abuse dynamics that the user cannot consciously recognize due to fragmentation.

\section{Implementation}
\label{sec:implementation}

\subsection{Repository Structure}

The \reunity{} implementation is available at: \texttt{github.com/ezernackchristopher97-cloud/ReUnity}

Key modules include:

\begin{itemize}
\item \texttt{src/reunity/core/entropy.py}: Shannon entropy, JS divergence, mutual information calculations
\item \texttt{src/reunity/router/state\_router.py}: Policy selection based on entropy state
\item \texttt{src/reunity/protective/pattern\_recognizer.py}: Harmful pattern detection
\item \texttt{src/reunity/memory/continuity\_store.py}: RIME implementation with consent scopes
\item \texttt{src/reunity/regime/regime\_controller.py}: Apostasis and regeneration logic
\end{itemize}

\subsection{Entropy State Detection}

The entropy analyzer classifies emotional states into five categories based on entropy values and stability metrics:

\begin{table}[H]
\centering
\caption{Entropy State Classification}
\label{tab:entropy_states}
\begin{tabular}{lccl}
\toprule
\textbf{State} & \textbf{Entropy Range} & \textbf{Lyapunov} & \textbf{System Response} \\
\midrule
STABLE & $H < 2.0$ & $\lambda < 0$ & Standard engagement \\
ELEVATED & $2.0 \leq H < 3.0$ & $\lambda \approx 0$ & Enhanced monitoring \\
HIGH & $3.0 \leq H < 4.0$ & $\lambda > 0$ & Supportive interventions \\
CRITICAL & $H \geq 4.0$ & $\lambda >> 0$ & Crisis protocols \\
SUPPRESSED & $H < 1.0$ & $\lambda << 0$ & Gentle activation \\
\bottomrule
\end{tabular}
\end{table}

\subsection{Protective Pattern Recognition}

The Protective Logic Module monitors for 13 harmful relational patterns:

\begin{enumerate}
\item Gaslighting (reality denial patterns)
\item Love bombing (excessive early affection)
\item Hot-cold cycles (intermittent reinforcement)
\item Isolation attempts
\item Financial control
\item Identity erosion
\item Blame shifting
\item Triangulation
\item Future faking
\item Trauma bonding indicators
\item Coercive control
\item Digital surveillance
\item Reproductive coercion
\end{enumerate}

Pattern detection uses a combination of linguistic analysis, temporal pattern matching, and entropy-based anomaly detection.

\subsection{Memory Architecture}

RIME stores identity fragments using a lattice-based memory graph constrained by divergence thresholds:

\begin{equation}
\text{Edge}(m_i, m_j) \text{ exists iff } JSD(m_i, m_j) < \theta_{max} \text{ and } I(m_i; m_j) > \theta_{min}
\label{eq:lattice}
\end{equation}

This ensures that connected memories share sufficient similarity (low divergence) and meaningful relationship (high mutual information) while preventing spurious connections.

\section{Experiments and Results}
\label{sec:experiments}

\subsection{Dataset}

We evaluated \reunity{} using the GoEmotions dataset from Google Research, containing 54,263 Reddit comments annotated with 27 emotion categories plus neutral. This dataset provides real human emotional expression with ground-truth labels for validation.

\subsection{Entropy Analysis Results}

Analysis of the GoEmotions dataset yielded the following entropy metrics:

\begin{table}[H]
\centering
\caption{Entropy Analysis Results (GoEmotions, n=54,263)}
\label{tab:entropy_results}
\begin{tabular}{lc}
\toprule
\textbf{Metric} & \textbf{Value} \\
\midrule
Mean Shannon Entropy & 4.01 bits \\
Maximum JS Divergence & 0.55 \\
Mean Mutual Information & 2.44 bits \\
Mean Lyapunov Exponent & 0.025 \\
\bottomrule
\end{tabular}
\end{table}

\begin{figure}[H]
\centering
\includegraphics[width=0.9\textwidth]{figures/simulation_1_entropy_analysis.pdf}
\caption{Entropy analysis results from GoEmotions dataset showing emotion distribution, Shannon entropy function, ReUnity state distribution, and entropy metrics comparison.}
\label{fig:entropy_results}
\end{figure}

\subsection{State Router Performance}

The state router classified 54,263 comments into \reunity{} states:

\begin{itemize}
\item STABLE: 64.6\% of comments
\item ELEVATED: 18.2\% of comments
\item HIGH: 12.1\% of comments
\item CRITICAL: 3.8\% of comments
\item SUPPRESSED: 1.3\% of comments
\end{itemize}

\begin{figure}[H]
\centering
\includegraphics[width=0.7\textwidth]{figures/simulation_5_state_router.pdf}
\caption{State router simulation results showing distribution across ReUnity states.}
\label{fig:router_results}
\end{figure}

\subsection{Pattern Detection Results}

The Protective Logic Module detected the following patterns in the GoEmotions dataset:

\begin{table}[H]
\centering
\caption{Pattern Detection Results}
\label{tab:pattern_results}
\begin{tabular}{lc}
\toprule
\textbf{Pattern} & \textbf{Detections} \\
\midrule
Hot-cold cycles & 231 \\
Gaslighting indicators & 187 \\
Isolation attempts & 143 \\
Blame shifting & 98 \\
Identity erosion & 67 \\
\bottomrule
\end{tabular}
\end{table}

\begin{figure}[H]
\centering
\includegraphics[width=0.7\textwidth]{figures/simulation_6_pattern_detection.pdf}
\caption{Protective pattern detection results from GoEmotions dataset.}
\label{fig:pattern_results}
\end{figure}

\subsection{Stability Analysis}

Lyapunov exponent analysis revealed:

\begin{itemize}
\item 72.3\% of emotional trajectories showed stable dynamics ($\lambda < 0$)
\item 19.4\% showed marginally stable dynamics ($\lambda \approx 0$)
\item 8.3\% showed chaotic dynamics ($\lambda > 0$)
\end{itemize}

\begin{figure}[H]
\centering
\includegraphics[width=0.9\textwidth]{figures/simulation_4_lyapunov_stability.pdf}
\caption{Lyapunov stability analysis showing distribution of exponents and stability classification.}
\label{fig:lyapunov_results}
\end{figure}

\section{Discussion}
\label{sec:discussion}

\subsection{Interpretation of Results}

The entropy analysis confirms that emotional states in natural language follow predictable information-theoretic patterns. The mean entropy of 4.01 bits indicates moderate emotional complexity, consistent with the diverse emotional content of Reddit comments.

The state router's classification demonstrates that the majority of emotional expressions fall within stable ranges, with a meaningful minority requiring enhanced support or crisis intervention. This distribution aligns with clinical expectations for general population samples.

Pattern detection results suggest that harmful relational dynamics are detectable in text data, though the relatively low detection rates reflect the general-population nature of the GoEmotions dataset rather than a clinical sample.

\subsection{Limitations}

Several limitations constrain the current implementation:

\begin{enumerate}
\item \textbf{Dataset Limitations:} GoEmotions contains general Reddit comments, not clinical samples from trauma survivors. Validation with clinical populations is needed.
\item \textbf{Linguistic Constraints:} Current pattern detection relies on English-language patterns; multilingual support requires additional development.
\item \textbf{Temporal Resolution:} The dataset lacks temporal metadata, preventing validation of trajectory-based predictions.
\item \textbf{Ground Truth:} No ground truth exists for ``correct'' entropy states or pattern detections; validation relies on face validity and clinical consultation.
\end{enumerate}

\subsection{Ethical Considerations}

\reunity{} is designed as a support tool, not a replacement for professional care. The system includes multiple safeguards:

\begin{itemize}
\item Crisis detection triggers referrals to human support, not automated intervention
\item All data remains under user control with full export and deletion capabilities
\item The system explicitly disclaims clinical or diagnostic functions
\item Pattern detection results are presented as observations, not judgments
\end{itemize}

\section{Conclusion}
\label{sec:conclusion}

This paper presented \reunity{}, an entropy-based AI framework for supporting identity continuity in individuals experiencing trauma-related fragmentation. The framework applies information-theoretic measures to emotional state analysis, enabling quantitative assessment of stability, transition dynamics, and relational patterns.

Empirical validation using the GoEmotions dataset demonstrates the framework's analytical capabilities, though clinical validation remains necessary. The open-source implementation provides a foundation for further development and adaptation to specific clinical contexts.

\reunity{} represents a fundamentally different approach to trauma-informed technology: rather than attempting to ``fix'' fragmentation, it provides tools for maintaining awareness and continuity across discontinuous states. This approach respects survivor autonomy while offering meaningful support during vulnerable periods.

Future work will focus on clinical validation, multilingual support, and integration with existing healthcare systems while maintaining the privacy-preserving, survivor-centered design principles that define the framework.

% ============================================================================
% APPENDICES
% ============================================================================
\newpage
\appendix

\section{Extended Social and Contextual Statistics}
\label{appendix:social}

This appendix provides comprehensive statistical data supporting the problem statement in Section~\ref{sec:introduction}.

\subsection{Montana Domestic Violence Statistics}

Montana women experience intimate partner violence at a lifetime prevalence rate of 56.7\%, significantly higher than many national averages. The state documented 248 domestic violence fatalities from 2000 to 2021, with 73\% of victims being female.

\begin{table}[H]
\centering
\caption{Montana Domestic Violence Statistics (2000-2021)}
\label{tab:montana_stats}
\begin{tabular}{lc}
\toprule
\textbf{Metric} & \textbf{Value} \\
\midrule
Total DV Fatalities & 248 \\
Female Victims & 73\% \\
Lifetime IPV Prevalence (Women) & 56.7\% \\
Counties Without Trauma Providers & 61\% \\
Average Rural Response Time & 45+ minutes \\
\bottomrule
\end{tabular}
\end{table}

\subsection{Rural Healthcare Access Barriers}

Rural communities face systematic barriers to domestic violence intervention:

\begin{itemize}
\item Geographic isolation requiring travel of 100+ miles for specialized services
\item Provider shortages with 61\% of Montana counties lacking trauma-informed psychiatric providers
\item Economic barriers limiting access to private transportation and lodging
\item Cultural factors emphasizing family privacy and community reputation
\end{itemize}

\subsection{Institutional Failure Case Studies}

\subsubsection{Montana State University}

Between 2018 and 2023, Montana State University received over \$2.3 million in federal Violence Against Women Act and Victims of Crime Act funding while engaging in documented patterns of retaliation against sexual assault survivors. Specific violations included manipulation of investigation timelines, witness coaching, weaponization of No-Contact Orders, and accessibility violations preventing disabled survivors from participating in proceedings.

\subsubsection{Darcy Buhmann Case}

Darcy Buhmann was murdered by her ex-partner in rural Montana after seeking help from multiple institutional systems. The case revealed systematic failures including delayed law enforcement response, lack of trauma-informed providers within 150 miles, legal system delays, and economic barriers limiting relocation options.

\section{Mathematical Derivations}
\label{appendix:math}

\subsection{Shannon Entropy Properties}

The Shannon entropy function $H(X) = -\sum p(x) \log_2 p(x)$ has the following properties relevant to emotional state analysis:

\begin{enumerate}
\item \textbf{Non-negativity:} $H(X) \geq 0$ for all distributions
\item \textbf{Maximum:} $H(X) \leq \log_2 n$ where $n$ is the number of states
\item \textbf{Concavity:} $H(\lambda P + (1-\lambda)Q) \geq \lambda H(P) + (1-\lambda)H(Q)$
\end{enumerate}

\subsection{Jensen-Shannon Divergence Derivation}

The JSD is derived from the Kullback-Leibler divergence:

\begin{align}
D_{KL}(P||Q) &= \sum_i P(i) \log_2 \frac{P(i)}{Q(i)} \\
M &= \frac{1}{2}(P + Q) \\
JSD(P||Q) &= \frac{1}{2}D_{KL}(P||M) + \frac{1}{2}D_{KL}(Q||M)
\end{align}

The JSD is symmetric ($JSD(P||Q) = JSD(Q||P)$) and bounded ($0 \leq JSD \leq 1$).

\subsection{Lyapunov Exponent Calculation}

For a discrete dynamical system $x_{n+1} = f(x_n)$, the Lyapunov exponent is:

\begin{equation}
\lambda = \lim_{N \to \infty} \frac{1}{N} \sum_{n=0}^{N-1} \log |f'(x_n)|
\end{equation}

In practice, we approximate this using finite time series with appropriate windowing.

\section{Implementation Details}
\label{appendix:implementation}

\subsection{Repository Structure}

\begin{verbatim}
ReUnity/
+-- src/reunity/
|   +-- core/
|   |   +-- entropy.py
|   |   +-- free_energy.py
|   +-- router/
|   |   +-- state_router.py
|   +-- protective/
|   |   +-- pattern_recognizer.py
|   |   +-- safety_assessment.py
|   +-- memory/
|   |   +-- continuity_store.py
|   |   +-- timeline_threading.py
|   +-- regime/
|   |   +-- regime_controller.py
|   +-- alter/
|   |   +-- alter_aware.py
|   +-- api/
|       +-- main.py
+-- tests/
+-- data/
+-- docs/
\end{verbatim}

\subsection{Reproduction Instructions}

\begin{verbatim}
git clone https://github.com/ezernackchristopher97-cloud/ReUnity
cd ReUnity
make setup
make sim-download-data
make sim-real
\end{verbatim}

\subsection{Configuration Parameters}

\begin{table}[H]
\centering
\caption{Default Configuration Parameters}
\label{tab:config}
\begin{tabular}{llc}
\toprule
\textbf{Parameter} & \textbf{Description} & \textbf{Default} \\
\midrule
entropy\_window & Sliding window size & 10 \\
jsd\_threshold & Maximum divergence for edges & 0.5 \\
mi\_threshold & Minimum mutual information & 0.1 \\
lyapunov\_window & Stability calculation window & 20 \\
crisis\_threshold & Entropy threshold for crisis & 4.0 \\
\bottomrule
\end{tabular}
\end{table}

\section{Additional Figures}
\label{appendix:figures}

\begin{figure}[H]
\centering
\includegraphics[width=0.8\textwidth]{figures/simulation_2_js_divergence.pdf}
\caption{Jensen-Shannon divergence matrix between ReUnity emotional states.}
\label{fig:appendix_jsd}
\end{figure}

\begin{figure}[H]
\centering
\includegraphics[width=0.8\textwidth]{figures/simulation_3_mutual_information.pdf}
\caption{Top 10 emotion co-occurrences showing mutual information between pairs.}
\label{fig:appendix_mi}
\end{figure}

\begin{figure}[H]
\centering
\includegraphics[width=0.9\textwidth]{figures/mathematical_foundations.pdf}
\caption{Mathematical foundations visualization showing entropy, divergence, and stability relationships.}
\label{fig:appendix_math}
\end{figure}

\begin{figure}[H]
\centering
\includegraphics[width=0.9\textwidth]{figures/recursive_consciousness_flow.pdf}
\caption{Recursive consciousness flow diagram showing information processing pathways.}
\label{fig:appendix_flow}
\end{figure}

% ============================================================================
% REFERENCES
% ============================================================================
\newpage
\printbibliography[title=References]

\end{document}
