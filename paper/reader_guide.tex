\newpage
\section*{How to Read This Document}
\addcontentsline{toc}{section}{How to Read This Document}

This document serves as the foundational whitepaper for the ReUnity platform and AI framework. It is designed for multiple audiences and does not need to be read linearly. Use this guide to navigate to the sections most relevant to your needs.

\subsection*{For Platform Overview Readers}
If you want to understand what ReUnity is and why it matters, read:
\begin{itemize}[noitemsep]
\item \textbf{Executive Summary} (Section 1): High-level overview of the framework and its purpose
\item \textbf{Problem Statement and Thesis} (Sections 2-3): The core argument and what ReUnity addresses
\item \textbf{AI Mirror System Architecture} (Section 6): How the system works at a conceptual level
\end{itemize}

\subsection*{For Survivors and Advocates}
If you are a survivor, advocate, or someone working directly with affected populations, focus on:
\begin{itemize}[noitemsep]
\item \textbf{Background and Problem Landscape} (Section 4): Understanding the systemic failures
\item \textbf{Clinical Applications and Specialized Protocols} (Section 10): How the system supports different trauma presentations
\item \textbf{Protective Logic: Pattern Recognition} (Section 9): How the system identifies harmful relationship patterns
\item \textbf{Extended Social and Contextual Statistics} (Appendix B): Case studies and lived experience context
\end{itemize}

\subsection*{For Developers and Researchers}
If you are building, extending, or validating the technical system, prioritize:
\begin{itemize}[noitemsep]
\item \textbf{Methodology} (Section 5): Mathematical foundations and information-theoretic approach
\item \textbf{Enhanced System Design and Technical Architecture} (Section 8): Implementation details
\item \textbf{Mathematical Derivations and Code Examples} (Appendix C): Complete code and equations
\item \textbf{Experimental Results} (Section 16): Empirical validation on GoEmotions dataset
\item \textbf{Data Sources and Provenance} (Appendix A): Reproducibility information
\end{itemize}

\subsection*{For Policy Makers and Institutions}
If you are focused on policy reform, funding, or institutional change, read:
\begin{itemize}[noitemsep]
\item \textbf{Economic Impact and Policy Analysis} (Section 15): Cost-benefit analysis and policy recommendations
\item \textbf{Institutional Capture and Grant Manipulation} (Section 1.1): Documentation of systemic failures
\item \textbf{Advanced Privacy and Security Protocols} (Section 11): Governance and data sovereignty frameworks
\end{itemize}

\subsection*{Content Mode Labels}
Throughout this document, content falls into five modes. Tone shifts between these modes are intentional:

\begin{description}[noitemsep]
\item[Technical Framework:] Mathematical foundations, algorithms, and system architecture
\item[Empirical Demonstration:] Validation results, dataset analysis, and measurable outcomes
\item[Policy Analysis:] Institutional critique, funding reform, and systemic recommendations
\item[Clinical Application:] Trauma-informed protocols, intervention approaches, and care integration
\item[Contextual Grounding:] Lived experience, case studies, and motivating context
\end{description}

\subsection*{A Note on Length}
This document is comprehensive by design. It serves as a permanent reference for the ReUnity platform, not a quick-read summary. The depth enables rigorous scrutiny while the navigation guide above enables efficient access to relevant sections.

\vspace{1cm}
\hrule
\vspace{0.5cm}
\textit{This document is not a clinical or treatment tool. It describes a research framework and technological platform. If you are in crisis, please contact emergency services or a crisis hotline.}

